\section{Taxis Protocol}
\label{sec:protocol}

In this section we present a new protocol \Taxis and its basic building blocks.
%
The ultimate product of \Taxis is a ledger \ledger providing fair transaction order.

Before we introduce \Taxis, we present its preliminary version \TaxisWL as a direct comparison with \textsf{Aequitas}.
%
\TaxisWL achieves consistency, weak-liveness and $(\varphi, \DBW)$-order-fairness.
%
Specifically, while the liveness is weak (same as \textsf{Aequitas}), this protocol achieves the best transaction order fairness that we can expect.
%
Next, by adding a few simple modifications on \TaxisWL, we present \Taxis that reconciles the tension between liveness and fair order.
%
The ledger of \Taxis satisfies consistency, (standard) liveness and $(\varphi, \TDBW)$-fair-order.

\Taxis is a two-stage protocol that decouples the mining procedure of profiles and the final serialization of transactions.
%
We will use blockchain as an intermediate information aggregator to collect profiles (i.e., transaction log) and build the ultimate ledger \ledger on top of this blockchain.
%
For simplicity, we present \TaxisWL and \Taxis assuming static number of participants and discuss how to adapt them to the dynamic participation in \cref{subsec:dynamic-participation}.

\paragraph{Blockchain notations.}
%
A block with target $T \in \mathbb{N}$ is a quadruple of the form $\block = \langle ctr, r, h, x \rangle$ where $ctr, r \in \mathbb{N}$, $h \in \{0, 1\}$ and $x \in \{0, 1\}^*$.
%
A blockchain \chain is a (possibly empty) sequence of blocks; the rightmost block by convention is denoted by \chainHead{\chain} (note $\chainHead{\varepsilon} = \varepsilon$).
%
These blocks are chained in the sense that if $\block_{i + 1} = \langle ctr, r, h, x \rangle$, then $h = H(\block_i)$, where $H(\cdot)$ is cryptographic hash function with output in $\{0, 1\}^\kappa$.
%
We adopt \timestamp{\block} to denote the timestamp of \block; and, slightly abusing the notations and omitting the current time \round, we will use $\chain^{\lceil k}$ to denote the chain from pruning all blocks \block such that $\timestamp{\block} \ge \round - k$.

\paragraph{2-for-1 proof-of-Work.}
%
2-for-1 PoW is a primitive that binds multiple PoW mining processes together by utilizing a single random oracle query.
%
It was first proposed in \cite{EC:GarKiaLeo15} to improve the corruption threshold in ledger consensus.
%
This primitive mitigates the possible attack with multiple independent mining processes, where the adversary can join forces to any one of the oracles and gain undesired advantage.

We will use 2-for-1 PoW to mine two types of blocks: ledger blocks and profile blocks.
%
Ledger blocks form the \Taxis blockchain and they will only include recent profile blocks (unlike regular blockchains, ledger blocks in \Taxis will not include any transactions ``directly'').
%
Meanwhile, parties will use profile blocks to report their local profiles.
%
We denote the mining target of ledger blocks and profile blocks by $T_\LB$ and $T_\PB$, respectively.
%
\Taxis will maintain a constant ratio between them; for simplicity, in our presentation and analysis, we assume $T_\LB = T_\PB$.

The main goal of adopting 2-for-1 PoW to bind the mining process of these two types of blocks together, is to achieve better \emph{chain quality}.
%
Recall that chain quality is bad in the Bitcoin backbone protocol \cite{EC:GarKiaLeo15,C:GarKiaLeo17}, where the adversary can contribute more blocks to the common prefix compared with her relative computational power.
%
By adopting 2-for-1 PoW, \Taxis guarantees that, for a sufficiently long time, $\varphi$ fraction of parties mine $\varphi$ fraction of the profiles (and they are all included by ledger blocks in the blockchain).

\paragraph{Freshness and recency parameter.}
%
For the sake of achieving better chain quality on profile blocks, certain changes should be made to the 2-for-1 mining procedure.
%
Ideally, the adversary \adv should not be allowed to mine profile blocks timestamped in the very future; and, blocks should go stale as time passes by so that \adv cannot choose to withhold them to gain a sudden advantage.
%
Analogous to the treatment in fruitchain \cite{PODC:PasShi17}, we introduce two mechanisms to help ensure the \emph{freshness} of profile blocks.
%
On one hand, the header of a profile block should point to the last block in the settled blockchain; this prevents the adversary from mining blocks in the very future, as an honest ledger block will introduce fresh randomness which is unpredictable for \adv.
%
On the other hand, we set a recency parameter $R$ (in rounds) such that a profile block \PB referring to a settled ledger block \LB will only be valid before time $\timestamp{\LB} + R$ (in other words, it cannot be included by a ledger block with timestamp later than $\timestamp{\LB} + R$).

\subsection{Taxis with Weak Liveness}
\label{subsec:taxis-weak-liveness}

\paragraph{Mining procedure.}
%
In every round, parties try to mine new blocks after they update their local chains according to the chain selection rule (see below for validation details).
%
Two different block contents will be prepared: ledger block content $\mathtt{LBContent}$ which contains all (valid) newly seen profile blocks; and profile block content $\mathtt{PBContent}$ that includes the local profile of the miner.
%
Parties then compute the Merkle root $\mathtt{st}_{\LB} = \mathsf{MerkleTree}(\mathtt{LBContent})$ and $\mathtt{st}_{\PB} = \mathsf{MerkleTree}(\mathtt{PBContent})$, respectively.
%
Next, miners make a single random oracle query with the following input: $ctr$, a random nonce; $h$, the reference to previous block; $h'$, the reference to the last block in the settled part; \round, the current timestamp; $\mathtt{st}_{\LB}$, the Merkle root of ledger block; and $\mathtt{st}_{\PB}$, the Merkle root of the profile block.
%
They receive an output
%
\[ u = H(ctr, h, h', \round, \mathtt{st}_{\LB}, \mathtt{st}_{\PB}). \]
%
If $u < T_\LB$, the party succeeds in mining a ledger block.
%
A new block \LB with content $\mathtt{LBContent}$ is generated and appended to the local chain.
%
If the value of the reversed output string (which we denote by \stringRev{u}) satisfies $\stringRev{u} < T_\PB$, a new profile block \PB is mined and will be diffused to the network.

Note that timestamp \round is shared information in both blocks, so it is impossible to get two products with different timestamps.
%
This prohibits the adversary from manipulating timestamp unless she completely drops from one mining procedure.
%
For a ledger block, the reference to the settled block ($h'$) and the Merkle root of profile blocks ($\mathtt{st}_{\PB}$) are dummy information and we do not care about their values, they are only useful when parties want to check their validity (see below).
%
Similarly, for a profile block, the reference to the previous block ($h$) and the Merkle root of ledger blocks ($\mathtt{st}_{\LB}$) are dummy information.

We also highlight that parties do not need to include their entire transaction log in \PB.
%
A prefix of the profile can be pruned if all transactions in that prefix appear in the settled blockchain for more than \PBWindowLen rounds (i.e., these transactions have been reported for sufficiently long time and parties agree on the set of transactions that precede them, see below for details).
%
Note that with \txDelay-disseminated transaction diffusion and chain-quality property of the blockchain, all transactions received by an honest party before time $t$ is guaranteed to be in the settled blockchain within a bounded amount of time (see protocol analysis).
%
Furthermore, if \party notices that its local transaction log shares a common prefix with another profile block \PB in the settled blockchain, then \party can produce profile blocks with pointer to \PB to indicate their common part and thus save space.

\begin{cccProtocol}
    {\TaxisWL-$\mathsf{MiningProcedure}$}
    {mining-procedure}
    {The mining procedure of \TaxisWL.}

    \begin{algorithmic}[1]
        \State{Fetch information from \funcDiffuse and $\funcDiffuse^{\mathsf{tx}, \txDelay}$ and get new chains $(\chain_1, \ldots, \chain_M)$, new transactions $(\tx_1, \ldots, \tx_i)$ and new profile blocks $(\PB_1, \ldots, \PB_j)$}

        \State{Set $\mathtt{buffer_\PB} \gets \mathtt{buffer_\PB} \concat (\PB_1, \ldots, \PB_j)$}

        \State{Set $\mathtt{localProfile} \gets \mathtt{localProfile} \concat ( \tx_1, \ldots, \tx_i )$} \label{code:mining-profiles}

        \State{$\chainLocal \gets \mathsf{maxvalid}(\chainLocal, \chain_1, \ldots, \chain_M)$}

        \State{$\mathtt{LBContent} \gets$ (valid) profile blocks in $\mathtt{buffer_\PB}$ that are not mined in \chainLocal}
        \State{$\mathtt{PBContent} \gets \mathtt{localProfile}$ \Comment{Remove a prefix to save space.}}

        \State{Set $h' \gets H(\chainHead{\chainLocal^{\lceil k}})$ and $h \gets H(\chainHead{\chainLocal})$}

        \State{Set $\mathtt{st}_{\LB} \gets \mathsf{MerkleTree}(\mathtt{LBContent})$ and $\mathtt{st}_{\PB} \gets \mathsf{MerkleTree}(\mathtt{PBContent})$}

        \State{$u \gets \mathsf{H}(ctr, h, h', \round, \mathtt{st}_{\LB}, \mathtt{st}_{\PB})$}

        \If{$u < T_\LB$}
        \State{Set $\LB \gets \langle ctr, h, h', \round, \mathtt{st}_{\LB}, \mathtt{st}_{\PB} \rangle$ and $\chainLocal \gets \chainLocal \concat \LB$}

        \State{Diffuse $\chainLocal, \mathtt{LBContent}$}

        \EndIf

        \If{$\stringRev{u} < T_\PB$}
        \State{Set $\PB \gets \langle ctr, h, h', \round,  \mathtt{st}_{\LB}, \mathtt{st}_{\PB} \rangle$}

        \State{Diffuse $\PB, \mathtt{PBContent}$}
        \EndIf

        \State{$ctr \gets ctr + 1$}
    \end{algorithmic}
\end{cccProtocol}

\paragraph{Validity check of chains.}
%
Recall that the \Taxis blockchain is similar to that of Bitcoin's (except that \Taxis includes additional 2-for-1 PoW information) and so we follow \cite{C:GarKiaLeo17} regarding the validity of ledger blocks.
%
In addition, we also need to check the validity of profile blocks.
%
For a valid profile block \PB, we require that its block header satisfies three properties:
%
(i) \PB correctly reports a reference to \LB where \LB is the last block after pruning the blockchain for $k$ rounds (where $k$ is the common prefix, see below);
%
(ii) \PB reports a timestamp that is earlier than the ledger block containing \PB but no later than $\timestamp{\LB} + R$;
%
and (iii) hash of \PB block header is smaller than the profile block target $T_\PB$.
%
A chain \chain in \Taxis is valid if \chain itself is valid and all the profile blocks included in \chain are valid.
%
See \cref{algorithm:is-valid-chain} for the complete description.

\begin{cccAlgorithm}
    {$\textsf{IsValidChain}(\chain, \round)$}
    {is-valid-chain}
    {The chain validation rule.}

    \begin{algorithmic}[1]
        \State{$r' \gets \round, \LB \gets \chainHead{\chain}$}
        \State{$\langle ctr, h, h', r, st_{\LB}, st_{\PB} \rangle \gets \LB$}
        \State{$h^* \gets H(ctr, h, h', r, st_{\LB}, st_{\PB})$}
        
        \While{$\chain \neq \varepsilon$}
        \State{$\langle ctr, h, h', r, st_{\LB}, st_{\PB} \rangle \gets \LB$}
        \State{$\mathsf{validLB}, \mathsf{validPB} \gets \true$}
        
        \LineComment{Check validity of ledger block}
        \If{$(h^* \neq H(\LB)) \vee (h^* \ge T) \wedge (ctr \ge 2^{32}) \vee (r \ge r')$}
        \State{$\mathsf{validLB} \gets \false$}
        \EndIf
        
        \LineComment{Check validity of profile blocks in \LB}
        \For{$\PB \in \LB$}
        \State{$\LB' \gets $ the last ledger block s.t. $\timestamp{\LB} \le \timestamp{\PB}- k$}
        \If{$\timestamp{\PB} \ge \timestamp{\LB} \vee h' \neq H(\LB') \vee \timestamp{\PB} \ge \timestamp{\LB'} + R \vee H(\PB) \ge T_\PB$}
        \State{$\mathsf{validPB} \gets \false$}
        \EndIf
        \EndFor
        
        \If{$\mathsf{validLB} \wedge \mathsf{validPB} = \true$}
        \State{$r' \gets r, h^* \gets h$}
        \State{Remove the rightmost block in \chain}
        \State{$\LB \gets \chainHead{\chain}$}
        \Else
        \State{\Return \false}
        \EndIf
        
        \EndWhile
        
        \State{\Return \true}
    \end{algorithmic}
\end{cccAlgorithm}

\paragraph{Extracting transaction order.}
%
We detail how the ledger \ledger is extracted in \TaxisWL.
%
Generally speaking, parties will use profile blocks in the settled part of the blockchain to build a dependency graph; then, transaction order is determined by running graph condensation and (possibly) \textsc{DirectedBandwidth} algorithm (see \cref{algorithm:directed-bandwidth}) on all SCCs.
%
Note that all of these computations can be done locally based on the on-chain information.

As protocol execution proceeds, local chains held by honest parties will share a long common prefix (we write $k$ as the common prefix parameter --- i.e., the rounds that parties need to prune their local chain).
%
Protocol participants will extract a transaction pool \txpool in their common prefix by selecting those transactions that have been reported for sufficiently long time.
%
More specifically, in order for a transaction \tx to be selected, there should exist a $K$-time-window of \tx, starting at time $t$ such that (i) $t$ is the timestamp of the earliest ledger block that includes a profile block \PB reporting \tx; and (ii) this $K$-time-window should be fully included in the settled blockchain --- i.e., at round \round a party only considers time window that starts before round $\round - k - K$.

Transactions in \txpool are then added to a dependency graph $G$ as vertices.
%
Regarding rules to add edges, for each transaction \tx we care about the profile blocks in its $K$-time-window: if the majority of these profile blocks report $\tx' \before \tx$, then we add a \emph{dotted} edge $(\tx', \tx)$ to $G$ (when $\tx'$ does not exist in $G$, a vertex of $\tx'$ is added).
%
Note that a dotted edge $(\tx', \tx)$ does not confirm the preference $\tx' \before \tx$ in $G$.
%
In order to count the edge in the subsequent computation, we need to wait for the $K$-time-window of $\tx'$ and see if the majority of those profile blocks report $\tx \before \tx'$.
%
When this holds, we update the dotted edge to \emph{solid} (all the subsequent computations on $G$ only consider solid edges).
%
The reason for designing this two-phase edge adding rule is because, for those transaction pairs such that no $\varphi$-preference holds, the adversary might be able to report conflicting orders in the corresponding $K$-time-windows\footnote{When an edge from $\tx'$ to \tx exist, \tx will not get confirmed into the ledger. Also note that, with overwhelming probability, solid edges will appear on those transaction pairs with $\varphi$-preference. For details, see the protocol analysis.}.

After constructing the dependency graph $G$, parties can linearize the transactions on top of $G$.
%
Notice that $G$ can be cyclic. Parties first compute the condensation graph $G^*$ of $G$ --- i.e. each SCC is replaced by a vertex.
%
Since $G^*$ is acyclic, there exist source vertices (i.e., vertices without incoming edges) in $G^*$.
%
Protocol participants do the following steps repeatedly.
%
Let $V_{\mathsf{source}}$ denote the set of all source vertices in $G^*$ such that for all $v \in V_{\mathsf{source}}$ all transactions in $v$ are in \txpool (transactions that are waiting for some unconfirmed ones will never be selected in $V_{\mathsf{source}}$).
%
If $V_{\mathsf{source}}$ is empty then parties terminate and output the final ledger \ledger.
%
Otherwise, they select $v \in V_{\mathsf{source}}$ such that the starting time of $v$'s associated $K$-time-window is the earliest among $V_{\mathsf{source}}$ (if a vertex in $G^*$ represents a SCC in $G$, we choose the earliest time window in that SCC).
%
Then, if $v$ represents a single vertex in $G$, parties append the corresponding transaction to \ledger directly; otherwise, they run \cref{algorithm:directed-bandwidth} to extract the bandwidth-optimal order $l$ on $v_\SCC$ (i.e., the component in the original graph that condenses to $v$ in $G^*$) and append $l$ to \ledger.
%
After processing $v$, we remove it from $G^*$ and this yields a new source vertex set $V_{\mathsf{source}}$.

We present the full serialization code in \cref{algorithm:taxiswl-extract-transaction-order}.
%
Note that we slightly abuse the notation of block timestamp and we write $\timestamp{\tx}$ as the beginning time of $K$-time-window associated with \tx --- i.e., it is the timestamp of the first ledger block \LB that includes a profile block with \tx.

\begin{cccAlgorithm}
    {\TaxisWL-$\mathsf{ExtractTransactionOrder}(\chainLocal, \round)$}
    {taxiswl-extract-transaction-order}
    {Extracting transaction order in \TaxisWL.}

    \begin{algorithmic}[1]
        \State{Initialize \txpool and graph $G$ to be empty.}
        \For{$\tx \in \PB \in \LB \in \chainLocal^{\lceil K + k}$}
        \State{Add $\tx$ to $\txpool$ and $v_\tx$ to $G$}
        \EndFor

        \LineComment{Build graph}

        \For{$\tx \in \txpool$}
        \State{$\PB_\tx \gets \{\PB \mathbin| \PB \in \LB$ s.t. $\timestamp{\tx} \le \timestamp{\LB} < \timestamp{\tx} + K\}$ }
        \For{$\tx' \in \PB \in \PB_\tx$}
        \If{majority of $\PB_\tx$ report $\tx' \before \tx$}
        \If{$(\tx', \tx) \in G$}
        \State{Mark $(\tx', \tx)$ as solid}
        \Else
        \State{Add edge $(\tx', \tx)$ to $G$ and mark as dotted}
        \EndIf
        \Else
        \If{$(\tx, \tx') \in G$}
        \State{Mark $(\tx, \tx')$ as solid}
        \Else
        \State{Add edge $(\tx, \tx')$ to $G$ and mark as dotted}
        \EndIf
        \EndIf
        \EndFor
        \EndFor\label{code:end-of-build-graph}

        \LineComment{Compute order.}
        \State{Compute the condensation graph $G^*$ of $G$.} \Comment{Using solid edges only}
        \State{Let $V_{\mathsf{source}}$ be the set of vertices in $G$ s.t. $v$ is a source vertex in $G^*$ and all transactions in $v$ are in \txpool.}

        \While{$V_{\mathsf{source}} \neq \emptyset$}

        \State{Let $v$ be the vertex in $V_{\mathsf{source}}$ with earliest \PBWindowLen-time-window}
        \State{$v_\SCC \gets$ component in $G$ that corresponds to $v$ in $G^*$}
        \State{Run \cref{algorithm:directed-bandwidth} on $v_\SCC$ and get transaction order $l$}
        \State{$\ledger = \ledger \concat l$}

        \State{Remove $v$ from $G^*$} \Comment{This yields a new $V_{\mathsf{source}}$}

        \EndWhile
    \end{algorithmic}

    \textsc{Output:} ledger \ledger (a list of transactions with strict total order).
\end{cccAlgorithm}

\paragraph{\TaxisWL ledger properties.}
%
With bounded dynamic participation and appropriate parameters, the ledger \ledger of \TaxisWL satisfies three properties --- consistency, weak-liveness and  $(\varphi, \DBW)$-order-fairness.
%
Note that for consistency, a suffix of \ledger should be pruned to be resistant to adversarial manipulation.
%
Refer to the protocol analysis in \cref{sec:security-analysis} and \cref{thm:consistency} to see the detailed consistency parameter.

\begin{theorem}[Informal]
    In a $(\gamma, s)$-respecting environment, if Conditions \eqref{condition:min-length}, \eqref{condition:error-absorb} and \eqref{condition:order-fairness-param} are satisfied, the ledger \ledger of \TaxisWL achieves consistency, weak-liveness and $(\varphi, \DBW)$-order-fairness except with probability negligibly small in the security parameter.
\end{theorem}

\subsection{Taxis with Standard Liveness}
\label{subsec:taxis-standard-liveness}

We present the full \Taxis protocol on top of \TaxisWL in this section.
%
Briefly speaking, we add a fallback mechanism to order transactions that remain unconfirmed for a long time based on the beginning point of their \PBWindowLen-time-window.
%
Note that we only make two simple changes in the mining and order-extraction stage.

\paragraph{Mining procedure.}
%
In \Taxis, parties book-keep the local receiving time of transactions; and, when mining profile blocks, they additionally attach these timestamps to each transaction.
%
I.e., we replace Line~\ref{code:mining-profiles} in \cref{protocol:mining-procedure} with ``Set $\mathtt{localProfile} \gets \mathtt{localProfile} \concat ( \langle \tx_1, \round \rangle, \ldots, \langle \tx_i, \round \rangle )$'' where $\round$ is party's current local time.
%
All the other steps in \cref{protocol:mining-procedure} remain the same.
%
Since parties will agree on the profiles of a transaction \tx in a sufficiently long time window, they will agree on the timestamp vector associated with \tx as well.

\begin{cccProtocol}
    {\Taxis-$\mathsf{MiningProcedure}$}
    {mining-procedure-taxis}
    {The mining procedure of \Taxis.}
    
    \begin{algorithmic}[1]
        \State{Fetch information from \funcDiffuse and $\funcDiffuse^{\mathsf{tx}, \txDelay}$ and get new chains $(\chain_1, \ldots, \chain_M)$, new transactions $(\tx_1, \ldots, \tx_i)$ and new profile blocks $(\PB_1, \ldots, \PB_j)$}

        \State{Set $\mathtt{buffer_\PB} \gets \mathtt{buffer_\PB} \concat (\PB_1, \ldots, \PB_j)$}

        \State{Set $\mathtt{localProfile} \gets \mathtt{localProfile} \concat ( \langle \tx_1, \round \rangle, \ldots, \langle \tx_i, \round \rangle )$}

        \State{$\chainLocal \gets \mathsf{maxvalid}(\chainLocal, \chain_1, \ldots, \chain_M)$}

        \State{$\mathtt{LBContent} \gets$ (valid) profile blocks in $\mathtt{buffer_\PB}$ that are not mined in \chainLocal}
        \State{$\mathtt{PBContent} \gets \mathtt{localProfile}$ \Comment{After removing a prefix in settled part}}

        \State{Set $h' \gets H(\chainHead{\chainLocal^{\lceil k}})$ and $h \gets H(\chainHead{\chainLocal})$}

        \State{Set $\mathtt{st}_{\LB} \gets \mathsf{MerkleTree}(\mathtt{LBContent})$ and $\mathtt{st}_{\PB} \gets \mathsf{MerkleTree}(\mathtt{PBContent})$}

        \State{$u \gets \mathsf{H}(ctr, h, h', \round, \mathtt{st}_{\LB}, \mathtt{st}_{\PB})$}

        \If{$u < T_\LB$}
        \State{Set $\LB \gets \langle ctr, h, h', \round, \mathtt{st}_{\LB}, \mathtt{st}_{\PB} \rangle$}

        \State{$\chainLocal \gets \chainLocal \concat \LB$}

        \State{Diffuse $\chainLocal, \mathtt{LBContent}$}

        \EndIf

        \If{$\stringRev{u} < T_\PB$}
        \State{Set $\PB \gets \langle ctr, h, h', \round,  \mathtt{st}_{\LB}, \mathtt{st}_{\PB} \rangle$}

        \State{Diffuse $\PB, \mathtt{PBContent}$}
        \EndIf

        \State{$ctr \gets ctr + 1$}
    \end{algorithmic}
\end{cccProtocol}

\paragraph{Extracting transaction order.}
%
During the order-extraction stage, a fallback mechanism is provided to deal with cycles that span for a long time.
%
Specifically, for all unconfirmed vertices $V_{\mathsf{unconfirmed}}$ in the condensation graph $G^*$, we check if there exist a vertex $v \in V_{\mathsf{unconfirmed}}$ such that its corresponding SCC $(v_\SCC)$ in $G$ contain transactions whose \PBWindowLen-time-window begins before $\round - (K + k + \varDelta_{\mathsf{timeout}})$\footnote{We note that two large cycles cannot run in parallel, and there is at most one vertex with multiple transactions that can pass the timeout check. Refer to protocol analysis for more details.}.
%
Note that $\varDelta_{\mathsf{timeout}}$ is a timeout parameter that indicates the cycle spans for a long time (see protocol analysis for more details).
%
If such $v$ in $G^*$ exists, we order all vertices in $v_\SCC$ in an increasing order based on their median timestamp.
%
For a transaction \tx, its median timestamp $\med(\tx)$ is computed on the timestamp vector associated with \tx in its \PBWindowLen-time-window.
%
Note that since parties will agree on \tx's timestamp vector, they will also agree on $\med(\tx)$; and, taking the median guarantees that $\med(\tx)$ falls in the \txDelay-dissemination time window with \tx, thus the results in \cref{thm:dbw-timestamp-order} applies.

In addition, when tracing the previous dependency graphs, \Taxis will be able to detect those large cycles by carefully comparing the beginning point of \PBWindowLen-time windows among all transactions in the cycle, so that it will process them using the same fallback mechanism (this guarantees consistency).

\begin{cccAlgorithm}
    {\Taxis-$\mathsf{ExtractTransactionOrder}(\chainLocal, \round)$}
    {taxis-extract-transaction-order}
    {Extracting transaction order in \Taxis.}

    \begin{algorithmic}[1]
        \State{Extract \txpool and build graph $G$} \Comment{Same as \cref{algorithm:taxiswl-extract-transaction-order} Line~1-\ref*{code:end-of-build-graph}}

        \LineComment{Compute order}
        \State{Compute the condensation graph $G^*$ of $G$} \Comment{Using solid edges only}
        \State{Let $V_{\mathsf{source}}$ be the set of vertices in $G$ s.t. $v$ is a source vertex in $G^*$ and all transactions in $v$ are in \txpool}

        \While{$V_{\mathsf{source}} \neq \emptyset$}

        \State{Let $v$ be the vertex in $V_{\mathsf{source}}$ with earliest \PBWindowLen-time-window}
        \State{$v_\SCC \gets$ component in $G$ that corresponds to $v$ in $G^*$}

        \If{$\exists v_1, v_2 \in v_\SCC$ s.t. $\timestamp{v_1} + \varDelta_{\mathsf{timeout}} \le \timestamp{v_2}$} \label{code:long-cycle-check}
        \State{Order transactions in $v_\SCC$ in an increasing order based on their median timestamps and get $l$.}
        \Else
        \State{Run \cref{algorithm:directed-bandwidth} on $v_\SCC$ and get transaction order $l$}
        \EndIf
        \State{$\ledger = \ledger \concat l$}

        \State{Remove $v$ from $G^*$} \Comment{This yields a new $V_{\mathsf{source}}$}
        \EndWhile

        \LineComment{Run fallback if a cycle lasts for long}
        \State{Let $V_{\mathsf{unconfirmed}}$ denote all vertices that remains in the graph}
        \For{$v \in V_{\mathsf{unconfirmed}}$}
        \State{$v_\SCC \gets$ component in $G$ that corresponds to $v$ in $G^*$}
        \If{$\exists v' \in v_\SCC$ s.t. $\timestamp{v'} \le \round - (K + k + \varDelta_{\mathsf{timeout}})$} \label{code:fallback-begin}
        \State{Order vertices in $v_\SCC$ in an increasing order based on their median timestamps and get a prefix $l$ up to $v'$}
        \State{$\ledger = \ledger \concat l$}
        \EndIf \label{code:fallback-end}
        \EndFor
    \end{algorithmic}

    \textsc{Output:} a list of transactions \ledger with strict total order.
\end{cccAlgorithm}

\paragraph{\Taxis ledger properties.}
%
We provide a full analysis of the security of \Taxis protocol with bounded dynamic participation in \cref{sec:security-analysis}.
%
Specifically, we prove that the ledger \ledger of \Taxis satisfies three desired properties --- consistency, (standard) liveness and $(\varphi, \TDBW)$-order-fairness.

\begin{theorem}
    In a $(\gamma, s)$-respecting environment, if Conditions \eqref{condition:min-length}, \eqref{condition:error-absorb} and \eqref{condition:order-fairness-param} are satisfied, the ledger \ledger of \Taxis achieves consistency, liveness and $(\varphi, \TDBW)$-order-fairness with parameters as specified in \cref{thm:consistency,thm:liveness} except with probability negligibly small in the security parameter.
\end{theorem}

\paragraph{Performance analysis of \Taxis.}
%
We detail the computation/communication complexity of the \Taxis protocol.
%
For the proof of work part and communication overhead, it requires a random oracle call per round and possibly (if a PoW is found) a message transmission with message size, worst case, linear in the security parameter plus the number of transactions that are disseminated within a sliding window of length polylogarithmic in the security parameter.

To maintain the local transaction dependency graph $G$, note that $G$ can be
built incrementally since all vertices and edges are extracted from the settled
part of the blockchain; and, every time a vertex \tx is added to $G$, the number
of computational steps required (which will add the necessary edges between the
vertices) is linear in the number of transactions that appear in \tx's
\PBWindowLen-time-window, which is also of length polylogarithmic in the
security parameter.

Regarding solving \textsc{DirectedBandwidth} on each SCC of the transaction dependency graph, note that while the exact algorithm (\cref{algorithm:directed-bandwidth}) from~\cite{FSTTCS:JKLSS19} consumes exponential time with respect to the number of concurrent transactions, we highlight that, in real execution, it runs in practical time for two reasons.
%
First, a polynomial-time fallback (Line~\ref{code:fallback-begin}-\ref{code:fallback-end} in \cref{algorithm:taxis-extract-transaction-order}) will be triggered after a time slack of length $\varDelta_{\mathsf{timeout}}$ has passed, where $\varDelta_{\mathsf{timeout}}$ is a parameter that is of the same order of magnitude with respect to the common prefix parameter (cf.~\cref{eq:ell-length,eq:window-length}), the size of input (i.e., the number of vertices in a SCC) to \textsc{DirectedBandwidth} is therefore bounded by a polylogarithmic function of $\kappa$ times the transaction throughput.
%
On the other hand, the transaction dependency graph of a large Condorcet cycle is of good structure\footnote{If a Condorcet cycle spans for a long time, and the time points that two transactions enter this system are sufficiently apart from each other, then the edge between these two transactions will never be selected as backward edge. For large Condorcet cycles, such type of edges account for the vast majority of all the edges. See a detailed explanation in \cref{subsec:exact-algorithm-over-dependency-graphs}.} such that we could improve the running time from $\bigO^\ast(3^n \cdot 2^{n^2})$ in \cite{FSTTCS:JKLSS19} to $f(t) \cdot n^t \cdot 2^{n t }$ where $t$ is the
transaction throughput and $f(t)$ is a function that depends only on $t$, note that $t \ll n$.
%
We present and analyze this algorithm in \cref{subsec:exact-algorithm-over-dependency-graphs}.
%
Also note that while this local computation is the most expensive step but
it only needs to be performed once for each SCC throughout the entire protocol execution.

\subsection{Taxis with Dynamic Participation}
\label{subsec:dynamic-participation}

We detail how to adapt the above two protocols \TaxisWL and \Taxis from static number of parties to the dynamic participation.

\paragraph{Difficulty recalculation function.}
%
In a permissionless environment, mining difficulties $T_\LB$ and $T_\PB$ should be adjusted as the participant population fluctuates.
%
We adopt the same difficulty adjustment function in Bitcoin\footnote{We refer to \cite{C:GarKiaLeo17} for more details on analysis of Bitcoin with dynamic participation.} to recalculate $T_\LB$ (so $T_\PB$ is adjusted as well).
%
More specifically, this function adjusts mining difficulty at the end of epochs (an epoch consists of $m$ blocks), and the target for the next epoch $T'_\LB$ is computed base on current target $T_\LB$ and the time elapsed to mine $m$ blocks --- i.e., $T'_\LB = \varLambda / (m / f) \cdot T_\LB$ where $f$ is the ideal block generation rate, and $\varLambda$ is the difference between the timestamps of the last block in the previous and current epoch.
%
In addition, the relative amount of difficulty that can be adjusted each time is bounded by a constant $\tau$ (which is similar to that in Bitcoin where $\tau = 4$ is in use).

\paragraph{Counting majority as accumulated difficulty.}
%
Note that after profile blocks are mined under different difficulties, we should count ``majority'' in a \PBWindowLen-time-window in a new sense.
%
Precisely, we say some blocks account for the majority if they accumulate more than $d / 2$ difficulty where $d$ is the total difficulty of all profile blocks in the time window.
%
This new majority rule applies to building graph stage in \cref{algorithm:taxiswl-extract-transaction-order,algorithm:taxis-extract-transaction-order}.
%
It also applies to the selection of median timestamps.
%
I.e., a timestamp $t$ is selected for \tx because $t$ is the earliest timestamp such that all profiles that report a timestamp no later than $t$ yields more than half of the accumulated difficulties in \tx's \PBWindowLen-time-window.

