\section{Further Related Works}
\label{sec:further-related-works}

In a long line of research on understanding order fairness in the state machine replica (SMR) problem, Schneider \cite{CSUR:Schneider90} first proposes \emph{order} --- ``Every non-faulty state machine replica processes the requests it receives in the same relative order'' --- as the third property (other than consistency and liveness) that an ideal BFT SMR protocol should satisfy.
%
This is later formalized by Garay and Kiayias \cite{RSA:GarKia20} as ``serializability'' in ledger consensus.
%
Serializability requires that if \tx enters all honest parties' mempool before $\tx'$, honest parties should reject the order $\tx', \tx$ (in their settled ledger).

We provide an overview of all existing works on defining order fairness and implementing fair-order protocols.
%
Roughly speaking, we classify them into three directions --- (i) blind-order-fairness, which orders transactions while hiding their content; (ii) block-order-fairness, solving the Condorcet paradox by claiming ``blocks''; and (iii) timed-order-fairness, defining a time-based fair-order without cyclic preferences.
%
We highlight that, compared with our contributions, none of these existing fair order notions provides bounded unfairness --- i.e., there is no bound in the definition on how ``unfairly'' a transaction could be put before another.

\subsection{Blind Order Fairness}

Blind order fairness \cite{TOPLAS:ReiBir94,C:CKPS01,ICNP:ACGLRTY18,IWQOS:SokRot20,TOKENOMICS:MalSza22,AFT:CDEG23} considers the fair-order problem in a practical point of view --- i.e., since transaction content is the main information that an adversary will use to manipulate transaction order, we hide the content before their order is finalized.
%
In other words, all protocol participants (including the adversary) are ``blind'' to the transactions; and, when parties learn the transaction content, it has already been too late to re-order them.
%
This concept has been proposed as ``input causality'' before the blockchain era, see \cite{TOPLAS:ReiBir94,C:CKPS01,DSN:DuaReiZha17} for more details.
%
For the sake of hiding content, cryptographic primitives are employed in blind-order-fairness protocols, e.g., commit-and-reveal \cite{ICNP:ACGLRTY18,IWQOS:SokRot20}, verifiable secret sharing and threshold encryption \cite{TOKENOMICS:MalSza22} and delay encryption \cite{AFT:CDEG23}.

Aside from blindness, some works provide additional ``fair'' (random) pending transaction selection rules.
%
\textsf{Helix} \cite{ICNP:ACGLRTY18} is a protocol that applies a verifiable random sampling on the public memory pool, thus selecting transactions with equal probability.
%
Analogous to this design, Sokolik and Rottenstreich \cite{IWQOS:SokRot20} present a random transaction inclusion scheme with weight measured by the time that transactions stay in the mempool.

Note that blind order fairness is orthogonal to receiver-side order fairness (see below).
%
They can be applied together to further improve the fairness of a system.

\subsection{Block Order Fairness}

Kelkar \textit{et al.} \cite{C:KZGJ20} identifies the existence of Condorcet paradox (which is first observed in the social choice theory) in defining fair-order based on aggregating $\varphi$-fraction of parties' individual preference.
%
Condorcet paradox shows that the collective preferences can be cyclic.
%
This non-transitivity directly leads to the impossibility of achieving receive-order-fairness where processors decide a final order following $\varphi$ fraction of honest parties' choice exactly (i.e., $\tx_1 \before^\varphi \tx_2$ indicates $\tx_1$ before $\tx_2$ in the output).

Towards the goal of mitigation, Kelkar \textit{et al.} work on a relatively weak definition, which blockifies all transactions in a Condorcet cycle and claims that they are processed simultaneously.
%
They call this $\varphi$-block-order-fairness --- ``If $\tx_1 \prec^\varphi \tx_2$, then the final output reports $\tx_1$ \emph{no later than} $\tx_2$.''.
%
We point out that the output of $\varphi$-block-order-fairness is actually a partial order --- i.e., it does not indicate the order inside a block.
%
Therefore in \cref{def:block-order-fairness}, we adopt $\orderOutput'$ to denote a surjection from $\mathbb{T}$ to $\{1, 2, \ldots, m\}$ where $m \le |\mathbb{T}|$.

\begin{definition}
    [$\varphi$-block-order-fairness \cite{C:KZGJ20}, restated]
    \label{def:block-order-fairness}
    
    A function $F$ satisfies \emph{$\varphi$-block-order-fairness} if for all input $\profileSet = \profile_1, \profile_2, \ldots, \profile_n$ and $\orderOutput' = F(\profileSet)$,
    %
    \[
        \tx_i \before^\varphi \tx_j \Longrightarrow \orderOutput'(\tx_i) \le \orderOutput'(\tx_j).
    \]
\end{definition}

\paragraph{\textsf{Aequitas} protocol family.}
%
Kelkar \textit{et al.} \cite{C:KZGJ20} also present a protocol family \textsf{Aequitas} that achieves $\varphi$-block-order-fairness and can tolerate the adversarial nodes for up to $1 / 2$ in a synchronous network and up to $1 / 4$ assuming asynchronous transaction dissemination.
%
Notably, $\varphi$-block-order-fairness in \textsf{Aequitas} is achieved by sacrificing (standard) liveness of the protocol.
%
An intuitive explanation is that when the protocol execution encounters Condorcet cycles, parties have to wait indefinitely until the end of these cycles (which can last forever in the worst case), thus blocking all the transactions in the cycle from getting settled.

Following the same definition in \cite{C:KZGJ20}, Kelkar \textit{et al.} \cite{ACCS:KDK22} proposes a variant of \textsf{Aequitas} that extends the participation model to a ``permissionless'' setting,
where (a small fraction of) parties can join and leave but the total number of parties running the protocol remains \emph{static} during the execution.
%
The general idea is to let the transaction order of a modular chain simulate the receiving order of a server in the previous model.
%
Then, parties will run the permissioned \textsf{Aequitas} protocol based on the on-chain data.
%
Note that this protocol only works when no Condorcet cycles exist (this is achieved by constraining the network delay).

\begin{remark}
    Kelkar \textit{et al.} \cite{C:KZGJ20,ACCS:KDK22} claim that \textsf{Aequitas} can achieve both standard liveness and $\varphi$-block-order-fairness when the transaction diffusion network is fully synchronous --- i.e., \tx is delivered to all honest parties within two consecutive rounds.
    %
    However, we point out that this is not true in our model where the input profiles are strict total orders on transactions. In other words, even if a bunch of transactions are received in the same ``round'', they may still form Condorcet cycles. Note that this is not a peculiarity of our model: any fine grain timing model would result on a strict total ordering for the  input profiles.
\end{remark}

In order to solve the weak-liveness issue in \textsf{Aequitas}, Kelkar \textit{et al.} propose a leader-based permissioned protocol \textsf{Themis} \cite{CCS:KDLJK23} that achieves \cref{def:block-order-fairness} with standard liveness.
%
This protocol is built on top of Hotstuff and is resistant to up to $1 / 4$ corrupted nodes.
%
\textsf{Themis} allows leaders to split transactions in a Condorcet cycle into batches and output transactions in each batch following the Hamiltonian-cycle order.
%
We discuss the downside of this ordering policy in the context of \cref{def:varphi-dbw-order-fairness} in \cref{subsec:order-fairness-from-directed-bandwidth}.

It is also worth noting that Cachin \textit{et al.} \cite{FC:CMSZ22} elaborate on a variant of $\varphi$-block-order-fairness, defining a differential order fairness property based on the standard validity notions for consensus protocols \cite{PODC:FitGar03}.
%
They also present an efficient atomic broadcast protocol that guarantees message delivery in a differential fair order.
%
Recently, Vafadar and Khabbazian \cite{AFT:VK23} further enhance block-order-fairness by adopting ranked-pairs to order transactions inside a Condorcet cycle, which they believe has good properties and can be useful to mitigate the Condorcet attacks.

\subsection{Timed Order Fairness.}

Block-order-fairness suffers from potential liveness failure unless assuming a non-corrupting adversary and order fairness parameter $\varphi = 1$.
%
However this would tremendously restrict the usage scenarios of the fair-order protocol.
%
Thus, some follow-up works \cite{AFT:Kursawe20,OSDI:ZSCZA20} focus on further weakening the order fairness notion to circumvent Condorcet cycles.

Since every protocol participants will conventionally maintain a local clock, it is reasonable to let them assign a local timestamp to each transaction.
%
Now, consider two (local) timestamp vectors of transaction $\tx, \tx'$ respectively.
%
We can decide their order as $\tx, \tx'$ if these two vectors are ``separated'' by a timestamp $\tau$.
%
More precisely, ``separate'' means that all timestamps assigned to \tx are earlier than $\tau$ and all timestamps assigned to $\tx'$ are later than $\tau$.

\begin{definition}
    [Timed-order-fairness \cite{AFT:Kursawe20,OSDI:ZSCZA20}, restated]
    \label{def:timed-order-fairness}

    A function $F$ satisfies \emph{timed-order-fairness} if for all input $\profileSet = \profile_1, \profile_2, \ldots, \profile_n$ and $\orderOutput = F(\profileSet)$,
    %
    \[
        \max \mapTimestamp(\tx_i) < \min \mapTimestamp(\tx_j) \Longrightarrow \orderOutput(\tx_i) < \orderOutput(\tx_j)
    \]
    %
    where \mapTimestamp is an admissible timestamp assignment with \profileSet.
\end{definition}

While working on the same direction, \cite{AFT:Kursawe20,OSDI:ZSCZA20} look at different problems. \cite{AFT:Kursawe20} designs a widget which consists of a fixed number of reputable validators to help any blockchain protocols ensure fair ordering; and \cite{OSDI:ZSCZA20} presents a state machine replica BFT protocol to achieve order fairness.

Timed-order-fairness has been criticized for strongly relying on synchronized clocks.
%
For instance, the definition becomes meaningless when local clocks are apart from each other for an hour.
%
Nonetheless, we highlight that ``timestamps'' in \cref{def:timed-order-fairness} does not necessarily need to be the real world time.
%
In fact, timed-order-fairness works as long as the protocol itself can maintain a ``protocol time'', which let all the time-aware parties' local clocks stays in a narrow interval.
%
This can be easily achieved with some clock synchronization protocols.

