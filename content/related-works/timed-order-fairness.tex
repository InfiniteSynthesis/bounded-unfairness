\subsection{Timed Order Fairness.}

Block-order-fairness suffers from potential liveness failure unless assuming a non-corrupting adversary and order fairness parameter $\varphi = 1$.
%
However this would tremendously restrict the usage scenarios of the fair-order protocol.
%
Thus, some follow-up works \cite{AFT:Kursawe20,OSDI:ZSCZA20} focus on further weakening the order fairness notion to circumvent Condorcet cycles.

Since every protocol participants will conventionally maintain a local clock, it is reasonable to let them assign a local timestamp to each transaction.
%
Now, consider two (local) timestamp vectors of transaction $\tx, \tx'$ respectively.
%
We can decide the their order as $\tx, \tx'$ if these two vectors are ``separated'' by a timestamp $\tau$.
%
More precisely, ``separate'' means that all timestamps assigned to \tx are earlier than $\tau$ and all timestamps assigned to $\tx'$ are later than $\tau$.

\begin{definition}
    [Timed-order-fairness \cite{AFT:Kursawe20,OSDI:ZSCZA20}, restated]
    \label{def:timed-order-fairness}

    A function $F$ satisfies \emph{timed-order-fairness} if for all input $\profileSet = \profile_1, \profile_2, \ldots, \profile_n$ and $\orderOutput = F(\profileSet)$,
    %
    \[
        \max \mapTimestamp(\tx_i) < \min \mapTimestamp(\tx_j) \Longrightarrow \orderOutput(\tx_i) < \orderOutput(\tx_j)
    \]
    %
    where \mapTimestamp is an admissible timestamp assignment with \profileSet.
\end{definition}

While working on the same direction, \cite{AFT:Kursawe20,OSDI:ZSCZA20} look at different problems. \cite{AFT:Kursawe20} designs a widget which consists of a fixed number of reputable validators to help any blockchain protocols ensure fair ordering; and \cite{OSDI:ZSCZA20} presents a state machine replica BFT protocol to achieve order fairness.

Timed-order-fairness has been criticized for strongly relying on synchronized clocks.
%
For instance, the definition becomes meaningless when local clocks are apart from each other for an hour.
%
Nonetheless, we highlight that ``timestamps'' in \cref{def:timed-order-fairness} does not necessarily need to be the real world time.
%
In fact, timed-order-fairness works as long as the protocol itself can maintain a ``protocol time'', which let all the time-aware parties' local clocks stays in a narrow interval.
%
This can be easily achieved with some clock synchronization protocols.
